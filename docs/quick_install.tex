To make it easier for new users, HestiaPi offers ready-to-burn image files for your SD card.

If you bought your HestiaPi with an SD card, skip this step.


\subsubsection{Prepare a new SD card}

With the image file downloaded, you need to use an image writing tool (we prefer Etcher from below links) to install it on your SD card. You can't simply copy-paste it. If you downloaded a ZIP version, unzip the .img file first before the next step.

For experienced Linux users, just use dd to write to the block device (SD card)
and use the conv=fsync to make sure everything gets written.  For a more
detailed set of instructions, choose the right guide for your system below
(courtesy of Raspberry Pi website -- thanks):

\begin{itemize}
\item \href{http://www.raspberrypi.org/documentation/installation/installing-images/linux.md}{Linux}
\item \href{http://www.raspberrypi.org/documentation/installation/installing-images/mac.md}{Mac OS}
\item \href{http://www.raspberrypi.org/documentation/installation/installing-images/windows.md}{Windows} (avoid if you can as people have reported issues flashing their card from Windows)
\end{itemize}

