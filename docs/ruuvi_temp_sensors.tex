The Ruuvi tags require a Ruuvi Gateway device in order to support MQTT. If you
do not have a Ruuvi Gateway, there is a project which will effectively make
one out of a Raspberry Pi here: \url{https://github.com/Scrin/ruuvi-go-gateway}

Unfortunately, the Ruuvi Gateway does not publish the data in a usable format
and the ruuvi-go-gateway outputs the data in that same unusable format in
order to maintain compatibility.  This mean a bridge is required to decode the
data and then output it in a usable format.  The code for the bridge can be
found at: \url{https://github.com/Scrin/RuuviBridge/}

Both of these can be installed on a Raspberry Pi to get the data into a usable
format. They can even be installed on the HestiaPi itself, if you're so
inclined. Consult the documentation for these projects on how to install and
configure them.

A typical configuration is to have the ruuvi-go-gateway publish the encoded
data to a local MQTT server and have RuuviBridge subscribe to that local MQTT
server and then publish the properly formatted data to the HestiaPi.

After you have ruuvi-go-gateway and RuuviBridge configured, the temperature
data will be going to your HestiaPi. The next step is to configure the HestiaPi
to use this data. See \ref{Remote Temperature Sensors} for more information on
how to do this.
