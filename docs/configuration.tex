\subsubsection{Default SSH Username and Password}

Username: pi

Password: hestia

SSH port: 22

\subsubsection{MQTT Configuration}
All the topics are defined in the .things
\href{https://github.com/HestiaPi/hestia-touch-openhab/wiki/File-Structure-&-Paths-ONE}{file}.

Confirm by subscribing from another laptop to all (\#) MQTT IDs and listen for
published messages while you play with your HestiaPi. For Linux users, run this
in a terminal:

\texttt{mosquitto\_sub -h [HESTIA\_PI\_IP] -d -t hestia/\#}

\subsubsection{How to Access My HestiaPi From Outside My House}

You will need a WiFi router with port forwarding feature (most routers do these
days) and if you don’t have a static IP (or if you don’t know what this is),
you will need to use a free Dynamic DNS service called
\href{https://www.noip.com/support/knowledgebase/install-ip-duc-onto-raspberry-pi/}{NoIP}.
Don’t worry -- although we can’t offer support on individual routers, we can
certainly point you in the right direction. Installation instructions on the
above link.  Alternatively you can use my.openhab.org which is a service hosted
externally and is not controlled by us or you but by OpenHAB itself. Go to:

\texttt{http://[YOUR-HESTIAPI-IP]:8080/paperui}

and select Add-ons > MISC and install ``openHAB Cloud Connector'' if not
installed. Once installed SSH into your HestiaPi (username: pi and password:
hestia) and type:

\texttt{cat /var/lib/openhab2/uuid}

write the output down. Then type:

\texttt{cat /var/lib/openhab2/openhabcloud/secret}

write this output down too.

Then go to \url{https://myopenhab.org} and create an account using your details
and the above information (UUID and secret). You can now access your HestiaPi
Touch from a browser or your mobile app (enter ``\url{https://myopenhab.org}''
as a remote url and your myopenHAB account username and password as
credentials).  The above steps are also available in youtube format
\href{https://www.youtube.com/watch?v=joz5f4ejJVc}{here} too.
