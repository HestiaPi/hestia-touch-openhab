\subsubsection{GPIO pin mappings}

\textbf{Relays}

\begin{tabular}{|c|c|c|}
\rowcolor{lightgray}
\hline
         & \textbf{Pin} & \textbf{GPIO} \\ 
\hline
 Relay 1 & 32           & GPIO12 \\ 
\hline
 Relay 2 & 16           & GPIO23 \\  
\hline
 Relay 3 & 12           & GPIO18 \\
\hline
 Relay 4 & 36           & GPIO16 \\
\hline
\end{tabular}

\textbf{Temperature sensor}

The BME or BMP sensor can be at either I2C address 0x76 or 0x77. The software
will check both addresses, in this order.

\begin{tabular}{|c|c|c|}
\rowcolor{lightgray}
\hline
     & \textbf{Pin} & \textbf{GPIO} \\ 
\hline
 SDA & 3            & GPIO02 \\ 
\hline
 SCL & 5            & GPIO03 \\ 
\hline
\end{tabular}

\textbf{LCD touchscreen pinout}

\begin{tabular}{|c|c|c|}
\rowcolor{lightgray}
\hline
\textbf{Pin no.} & \textbf{Symbol} & \textbf{Description} \\ 
\hline
1, 17 & 3.3V & Power positive (3.3V power input) \\ 
\hline
2, 4 & 5V & Power positive (5V power input) \\ 
\hline
3, 5, 7, 8, 10, 12, 13, 15, 16 & NC & Not connected \\ 
\hline
6, 9, 14, 20, 25 & GND & Ground \\ 
\hline
11 & TP\_IRQ & Touch Panel interrupt, low level while the Touch Panel detects touching \\ 
\hline
18 & LCD\_RS & Instruction/Data Register selection \\
\hline
19 & LCD\_SI/TP\_SI & SPI data input of LCD/Touch Panel \\
\hline
21 & TP\_SO & SPI data output of Touch Panel \\
\hline
22 & RST & Reset \\
\hline
23 & LCD\_SCK/TP\_SCK & SPI clock of LCD/Touch Panel \\
\hline
24 & LCD\_CS & LCD chip selection, low active \\
\hline
26 & TP\_CS & Touch Panel chip selection, low active \\
\hline
\end{tabular}


\subsubsection{Configuration files}

\textbf{WiFi details}

\texttt{/etc/wpa\_supplicant/wpa\_supplicant.conf}

\textbf{OpenHAB}
Items

\texttt{/etc/openhab2/items/default.items}

Rules

\texttt{/etc/openhab2/rules/default.rules}

Sitemaps

\texttt{/etc/openhab2/sitemaps/default.sitemap}

Things

\texttt{/etc/openhab2/things/default.things}

Logs

\texttt{/var/log/openhab2/events.log\\
/var/log/openhab2/openhab.log}

\textbf{LCD UI}
The LCD UI is an HTML-based page loaded on a fullscreen browser. All HTML, CSS, JS, fonts and icon files are in here

\texttt{/home/pi/scripts/oneui}

The vue framework is used.
    
\textbf{Scripts}
In 
\texttt{/home/pi/scripts}

There are
\texttt{AdafruitDHTHum.py\\
AdafruitDHTTemp.py}

Read sensor data from DHT sensors.

\texttt{C2F.sh\\
F2C.sh}

Change HestiaPi from Celcius to Fahrenheit and vice versa.

\texttt{getBMEhumi.sh\\
getBMEtemp.sh\\
getBMEpress.sh}

Read sensor data from BME sensors (calling bme280.py).

\texttt{getcputemperature.sh}

Returns RasPi CPU temperature.

\texttt{getssid.sh}

Returns WiFi SSID name.

\texttt{gettz.sh}

Returns system Timezone.

\texttt{getuseddiskspace.sh}

Returns used SD card space.

\texttt{getwifiinfo.sh }

Returns WiFi signal strength.

\texttt{getwlan0ip.sh }

Returns WiFi IP.

\texttt{getwlan0mac.sh }

Returns WiFi MAC address.

\texttt{netcheck.sh }

Cron script that checks WiFi connectivity by pinging its gateway. If no
response is received at the first time, the WiFi interface is restarted and a
DHCP (dynamic) IP is requested. If no response is received again RaspberryPi,
the reboot command is sent. Please note this script is not enabled by default
and you will need to follow the instructions supplied at the top of the file.
Please also note that restarting the Pi will stop any current task and will not
resume after restart.

\texttt{openhabloader.sh }

Loads the Touch LCD UI.

\texttt{getpublicip.sh }

Checks current public IP and if it matches with previous reading, it does
nothing else. If current public IP is different, the latest value is sent to
your account (manual and free account registration needed).
    
\textbf{Web UI}

\texttt{http://[YOUR\_HESTIA\_IP]:8080/basicui/app}

or simply


\texttt{http://[YOUR\_HESTIA\_IP]:8080}

and then select Basic UI and default

\textbf{Smartphone App}
Under Settings > Local server settings


\texttt{http://[YOUR\_HESTIA\_IP]:8080}
