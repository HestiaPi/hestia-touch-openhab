By default, the HestiaPi does not connect to any external servers. If you want
the thermostat to tell a dynamic DNS service about the its publis IP address,
the
\href{https://github.com/HestiaPi/hestia-touch-openhab/blob/ONE/home/pi/scripts/getpublicip.sh}
{getpublicip.sh} script can help you do just that.

The file is already on your HestiaPi at:

\texttt{/home/pi/scripts/getpublicip.sh.bak}

You will need an account with whatever service you choose to use inside the
script, and the will need to be modified to specify your username, password
and possibly domain name. There are some example in the script near the bottom
which are commented out. Customize and uncomment them and then rename the
script to \texttt{getpubliship.sh} and you should be good to go. This file is
already run periodically by OpenHAB.

Note: Old versions of the HestiaPi software automatically reached out to
ipinfo.io to determine your public IP address. For these versions, the script
is already named \texttt{getpublicip.sh} and so you just have to edit that
file to get it to report the public IP to your dynamic DNS provider.
