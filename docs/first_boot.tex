Fix your HestiaPi's case to the wall first. If you simply want to test-drive
HestiaPi before committing to it, connect the LCD first and then plug in a
Micro USB cable to the Pi's port.

\begin{enumerate}
\item Insert the MicroSD card back in the Raspberry Pi. Just push it in. It
      does not click. It does not lock in place. A tiny part of it will stick
      out just enough to grab and pull it if needed.
\item Insert the LCD in the cover. Remove the protective film if present. Turn
      and push it in place. It should feel firm in place.
\item Take all necessary precautions before applying mains voltage so cut off
      the power now!
\item Connect all Heating, Cooling, Fan and Hot Water (depending on model)
      control lines on the terminal block's contacts.
\item Connect remaining wires, marked C(N) and R(L).
\item Place the sensor at the bottom compartment of the cover and fit the 4
      wires in the vertical slit. Note that the sensor, the little shiny
      square, should be placed facing outwards and ideally not be blocked by
      any plastic piece of the case. The red wire (Vin) goes to the top pin
      (Vin) on the PCB.
\item Align and push evenly the cover against the base aligning at the same
      times the pins with the LCD connector. The cover should lock when pushed
      all the way in. Step back and enjoy the new looks of your wall :)
\item If you cannot cut off the power on the cables, you are risking of
      HestiaPi booting before the LCD is connected. In such a scenario the LCD
      will not display anything but a blank white screen and you would need to
      restart as it is not ``plug and play'' like HDMI. We would advise to leave
      the SD card out before applying mains voltage and just before you are
      about to close the case, insert it but don't restart. It shouldn't boot.
      Once you close the case, there is a chance that it will restart. Close
      the case and wait 20 seconds. If nothing shows up on the screen, it
      didn't restart.  Press reset button from the right side.
\item If at any time you want to remove the top cover, select ``Shutdown'' from
      the App. When HestiaPi Touch is completely shut down, simply pull the
      cover outwards.
\item You should soon see the HestiaPi boot sequence and the loading screen at
      the end with a countdown. Follow
      \href{https://github.com/HestiaPi/hestia-touch-openhab/wiki/Connect-WiFi}
      {these steps} to connect your new HestiaPi to your WiFi.
\item After a few seconds the screen will show if the WiFi is connected and
      what is the local IP it got (DHCP)
\item The full installation may take up to 20 minutes for the very first time
      and a few restarts are normal. Just leave it alone. You can always SSH to
      it.  Use pi/hestia
\item The SD card image expands automatically to occupy the complete size of
      the card if available.
\item While waiting, head over to the \href{https://hestiapi.com/downloads/}{downloads}
      section and download the smartphone app on your phone. Under settings set
      Local OpenHAB URL as http://[hestiapi\_IP]:8080 and close the application
\item The LCD UI starts with 0 values or blank fields. This is normal until it
      gets ready.
\item Once the LCD is showing the UI with temperature values, try and load the
      app again or simply use your laptop and navigate to:\\
      http://[hestiapi\_IP]:8080/start/index and select ``Basic UI''
\item You should now be able to control the basic functions from either the App
      or your browser
\item Please note that the UI of the app, web and LCD may change with software
      updates so back up your customisations before running an update.
\item OpenHAB2 has a great \href{https://community.openhab.org/}{forum} with so
      much information from fellow users. Salivate at what you want to make now
      with it.
\item Feel free to explore the files under /etc/openhab2 names default.* in
      folders items, rules, sitemaps and things.
\end{enumerate}
