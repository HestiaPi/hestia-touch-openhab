This section will help you determine if your heating, venting and air
conditioning (HVAC) system is compatible with a HestiaPi. HVAC standards vary
around the world, but the HestiaPi has been designed to be compatible with the
standard systems in the United States of America and in Europe.

It is important to note that while the vast majority of systems use the
standard controls for the country they are in, some have proprietary
controls. The sections below will help you determine if your home uses the
standard HVAC controls.

If you have additional wires that are not listed in the sections below, the
HestiaPi may not be able to control all the functionality of your system. You
will want to check the owner's manual for your HVAC system to determine what
the additional wires do and whether you would be comfortable with losing that
functionality.

\subsection{USA}
If you open your existing thermostat and see wires labeled R, G, W, and Y, then
your system is compatible.

If you also have a wire labeled C, then the HestiaPi can be powered by the HVAC
system, eliminating the need to plug the thermostat into an outlet. A wiring
diagram can be found in Figure \ref{fig:us}.

If you have two stage heating, which is more common with heat pumps than gas,
oil, or electric furnaces, you should also see a W2 wire. This is also
supported by the HestiaPi.

\subsection{Europe}
If you open your existing thermostat and see wires labeled N, Hum, W, and H then
your system will be compatible with the HestiaPi and can be powered directly
from the HVAC system, eliminating the need for plugging the thermostat into a
power outlet.  A wiring diagram can be found in Figure \ref{fig:eu}.
