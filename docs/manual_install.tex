For people who want to install everything step-by-step, or just see how it is
done, the process is automated in
\url{https://gitlab.hax0rbana.org/public-repos/raspberrypi-automation/}.

That repo is not specific to the HestiaPi, it automates building multiple
Raspberry Pi images that someone might want for their projects. The main file
that is relevant to the HestiaPi project is hestiapi.yml.

The README file in that repo has the latest information on exactly how to build
the image, but the general process to build this at the time of writing is as
follows:

\begin{verbatim}
# Install the necessary software
apt-get update && apt-get install -y ansible xz-utils ansible

# Get a fully patched Raspberry Pi image
PLAYBOOK=upgrade.yml ./build_emulator.sh arm1176 bullseye wrapper_ansible.sh

# Configure that image to be a HestiaPi
PLAYBOOK=hestiapi.yml EXTRA_VARS="hestiapi_version=v1.4-dev" \
        ./build\_emulator.sh arm1176 bullseye wrapper_ansible.sh

# At this point you will have your HestiaPi image at the following location
ls -l qemu-rpi/2022-09-22-raspios-bullseye-armhf-lite.img
\end{verbatim}

These instructions were tested on Ubuntu and should work on any Debian-based
Linux distribution.
