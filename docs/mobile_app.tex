The interface for the mobile app is almost identical to the basic UI of the
webpage (covered in section \ref{Webpage}).  The only signficiant differences
are getting the application and connecting to the HestiaPi.

The application can be downloaded from
\href{https://f-droid.org/en/packages/org.openhab.habdroid/}{F-Droid}, the
\href{https://play.google.com/store/apps/details?id=org.openhab.habdroid&hl=en_US}
{Google Play store}, or \href{https://apps.apple.com/us/app/openhab/id492054521}
{Apple's App store}. Ideally, as long as your mobile device is connected to
the same network as the HestiaPi, the app should automatically find the
HestiaPi's OpenHAB server. If this works as expected, everything should look
similar to the screenshots shown in figure \ref{fig:Web UI}.

If the server is not found, the hamburger menu in the top left (three
horizontal lines) will bring up a menu that allows access to the settings.  In
the settings menu, there is a Local section which allows connecting to a local
OpenHAB server.  The app refuses to connect to unencrypted web servers when the
URL is entered manually, so the URL should be slightly different than described
in the web UI: \texttt{https://[YOUR\_HESTIA\_IP]:8443/}  Once the URL is
entered, return to the settings screen and in the local section, it should say
``Insecurely connected to \texttt{YOUR\_HESTIA\_IP}''.  It says the connection
is insecure because the app has no way to verify that the server is actually
the correct one.  The app will work fine when connected ``insecurely'' and to
get a secure connection requires quite a bit of effort and technical know-how.
For instructions on how to get the app to say it's a secure connection, see
the ``\nameref{Set up TLS}'' section (\ref{Set up TLS}).

Now that the application is connected to the server, click the back arrow to
return to the Main Menu.
