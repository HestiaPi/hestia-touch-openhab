\subsubsection{Taps do not register on the touchscreen}
If the screen doesn't turn on, see the entry below about that. These steps are
to troubleshoot situations where the display is working correctly, but just not
responding to touches.

This most often is caused by a pinched screen. This can be either the tabs of
the case pinching the screen too tightly or the case pinching down on the front
of the screen. First, look for any wires that may have gotten pinched between
the relays or power supply (the tall components) and the screen when the case
was snapped into place. This is the most common cause of this issue and
fortunately it is easily resolved.

To confirm that it is a pinch, connect the screen without the shell of the case
and power on the HestiaPi. If it works correctly, this tells us that the case
is to blame.

To determine if it's the shell pinching down when it is closed, remove the
HestiaPi from the case's backplate, then connect the screen while it is in the
case's shell. Do not snap on the backplate. If the touchscreen works in this
configuration, we know the tabs that hold the LCD to the shell are not the
problem.

If the screen does not work with the backplate off, we know that the tabs are
pinching the screen too tightly. In this case, the tabs can be very carefully
scraped down with a hobby knife, box cutter, file or something of this nature.
Take care not to break of the tabs or hurt yourself.

If you are in a situation where the shell of the case is pinching down on the
screen only when it's connected with the backplate (and there are no wires
getting in between the screen and the rest of the HestiaPi), it means the case
is a bad fit. One option is to not quite snap the case completely shut. The LCD
pins will hold the shell in place and it should work fine. The next option is
to run without the shell on, which will give you a very punk look to your
thermostat. Or you can get a custom printed shell for your case that is slightly
taller.

If you are an advanced user and want to try to try to resolve this issue, the
tails of the headers on the back of the main PCB can sometimes be trimmed to
get just a tiny bit more room, which is often all that is needed. The PCB
should generally be as close to the backplate as possible. Another possibility
is to trim all of the LCD headers to make them just a little bit shorter. This
will make it even more difficult to align the pins when putting the case on,
but people have successfully repaired their thermostats by doing this.

Other causes of the touchscreen not registering taps include: the Raspberry Pi
was powered on when the LCD screen was connected improperly, someone pushed on
the screen too hard, or only some of the pins are fully connected to the
screen. If the touch part of the screen is broken, the above tests will all
have the same results (no response to taps). At this point, your options are to
replace the screen or just use the screen for viewing information and make all
the changes to the thermostat over wifi.

\subsubsection{Screen doesn't turn on}
If you see lights illuminated on the pi, but the screen doesn't turn on at all,
the screen is not receiving power, which means it is likely not connected
properly. Look in through the side of the case and see if you can tell if the
pins are aligned correctly. If you can not determine this, try removing the
shell of the case and connect the screen on without the shell of the case. The
screen should light up slightly when powered on. If this still doesn't work, it
is likely a defective screen. If you have another Raspberry Pi available, you
can connect it to the top left pins of the Pi (when the SD card is on the left)
and see if the screen lights up when that Pi is powered on with the HestiaPi's
SD card in it.

If the screen lights up slightly, but is blank, this is a different issue. In
this case, the screen is likely working and it's an issue with the software.
You should see the screen flicker within 60 seconds and you should see the
boot messages after a few minutes.

The fastest and easiest fix is to just copy the image onto the SD card again
and try again. If you are an advanced user and want to spend time
troubleshooting the issue to determine the root cause, we encourage you to
share whatever you find with the community at
\url{https://community.hestiapi.com}.

\subsubsection{Screen is slow to respond to taps}
The HestiaPi should respond to touches within a second. If it's responding
to taps, but it taking a long time, it's likely a software issue. In this
case, copying a fresh image onto the SD card will likely fix it. The other
thing that can be done to get it to respond faster to touches is to get the
fastest SD card available.

A video has been made to demonstrate how quickly the screen should respond to
taps. A stylus is used here to make it clear when the tap is occurring versus
when the screen is updating, but it should work the same when tapping with your
finger. \url{https://peertube.gsugambit.com/w/1AhKQByTrAg38HKSQUJFR3}

\subsubsection{How to edit files via SSH}
If you are very new to command line interface we would advise you taking a
short online course by searching for ``linux command line interface'' on your
favourite website.

To edit a file while you are inside SSH use the command

\texttt{sudo nano /path/to/your/file}

Then leave your mouse alone as it does not control you cursor anymore

Use only your keyboard and once you are done, press Ctrl+O to save and Ctrl+X to close.

\subsubsection{Start OpenHAB2 in Debug Mode}

For OpenHAB2 (v10.x image -- July 2018)
To monitor the OpenHAB logs without stopping the service run

\texttt{openhab-cli showlogs}

To start OpenHAB manually after stopping the service run

\texttt{openhab-cli start}

For older OpenHAB installations:
Stop OpenHAB first

\texttt{sudo service openhab2 stop}

and when it is stopped, start it manually

\texttt{/usr/share/openhab2/start\_debug.sh}

once (if) loaded type inside the OpenHAB session

\texttt{log:tail}

and notice any issues.
