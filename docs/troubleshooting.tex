\subsubsection{General issues}
For those who are having problems and don't want to read the entire
troubleshooting section, just try reflashing the SD card with the latest image
from \url{https://hestiapi.com/downloads}. This really will fix nearly every
problem! So if your issue isn't listed, or you just want a quick fix, re-flash
that SD card and you'll likely be back in business in no time.

\subsubsection{Verifying relay status}
If you are unsure if the relays are turning on or off when they should, this
can be checked in software. To do so, you will need to be able to SSH into the
HestiaPi.

To determine if a relay on, SSH into the pi and then look at the GPIO value.
The mappings of which GPIO pins are connected to which relays are in section
\ref{GPIO pin mappings}.

As an example, relay 1 is GPIO12, which is the fan for HVAC systems and
humidity when in Generic (EU) mode. After SSHing into the HestiaPi, the command
below would display a ``1'' if that function was on, or a ``0'' if it was off.

\texttt{cat /sys/class/gpio/gpio12/value}

If this reports that the relay is on, but the the fan (or humidity for EU
systems) is not on, it indicates that there's likely an issue with wiring or
with your HVAC system.

If this reports that the functionality is off, when you think it should be on,
you may want to manually turn it on. After SSHing into the HestiaPi, you will
need to become root, and then set the GPIO value. To continue with the fan
example on HVAC systems, here are commands you would run after SSHing into the
pi in order to turn the fan on, and then turn it back off again.

\texttt{sudo su -\\
echo 1 | /sys/class/gpio/gpio12/value \# turn on\\
echo 0 | /sys/class/gpio/gpio12/value \# turn off\\
}

As soon as you run the ``echo 1'' command, the fan should turn on. There should
not be any delay. The same should be true when running the command to turn off
the relay.

Turning on relays in this manner bypasses the thermostat logic, which means
that the function will never turn off. That's why it's important to make sure
to manually turn off the functionality after you manually turn it on. If you
forgot to turn off the cooling system after manually turning it on, it'd likely
burn out the compressor, which would be a very expensive repair.

Running these commands should help you in determining if there is an issue with
the hardware or with the software. If these command do not cause the functions
to turn on, it points to a hardware issue, such as incorrect wiring.

If these commands work as expected but the heat does not turn on when the
heating set point is higher than the temperature, and the heating system is
turned on, it indicates that something is going wrong in the software (more
specifically, in OpenHAB). If that's the case, you can get help on the forum
at \url{https://community.hestiapi.com/}. Please be patient, as we are all
volunteers and it may be a few days before we see your message and have time to
investigate and reply.

\subsubsection{Eternal loading screen}
If your thermostat ever reboots and the loading countdown on the screen goes
all the way down to the point where it just says ``Loading...'' and stays there
for hours, it's either a software issue or a slow micro SD card.

First, reflash the SD card and this will almost certainly have you back up and
running.

If that does not work, the only other potential cause we have seen is using an
old, generic, or slow SD card. Using a card rated at 10MB/sec should be
sufficient, and 30MB/sec would be ideal. Unlabeled cards, or ones rated at
lower speeds may work. We have not found a card that will consistently cause
this issue, so we can't say for sure what the minimum acceptable speed is.

An icon that looks like a 10 inside a circle, a 1 inside a letter U, or a V10
are all icons that mean it's 10MB/sec. A 3 inside a U, or V30 indicated it is
a 30MB/sec card. There are faster cards, labeled V60 and V90, but getting a
card that fast is not necessary.

\subsubsection{Taps do not register on the touchscreen}
If the screen doesn't turn on, see the entry below about that. These steps are
to troubleshoot situations where the display is working correctly, but just not
responding to touches.

This most often is caused by a pinched screen. This can be either the tabs of
the case pinching the screen too tightly or the case pinching down on the front
of the screen. First, look for any wires that may have gotten pinched between
the relays or power supply (the tall components) and the screen when the case
was snapped into place. This is the most common cause of this issue and
fortunately it is easily resolved.

To confirm that it is a pinch, connect the screen without the shell of the case
and power on the HestiaPi. If it works correctly, this tells us that the case
is to blame.

To determine if it's the shell pinching down when it is closed, remove the
HestiaPi from the case's backplate, then connect the screen while it is in the
case's shell. Do not snap on the backplate. If the touchscreen works in this
configuration, we know the tabs that hold the LCD to the shell are not the
problem.

If the screen does not work with the backplate off, we know that the tabs are
pinching the screen too tightly. In this case, the tabs can be very carefully
scraped down with a hobby knife, box cutter, file or something of this nature.
Take care not to break of the tabs or hurt yourself.

If you are in a situation where the shell of the case is pinching down on the
screen only when it's connected with the backplate (and there are no wires
getting in between the screen and the rest of the HestiaPi), it means the case
is a bad fit. One option is to not quite snap the case completely shut. The LCD
pins will hold the shell in place and it should work fine. The next option is
to run without the shell on, which will give you a very punk look to your
thermostat. Or you can get a custom printed shell for your case that is slightly
taller.

If you are an advanced user and want to try to try to resolve this issue, the
tails of the headers on the back of the main PCB can sometimes be trimmed to
get just a tiny bit more room, which is often all that is needed. The PCB
should generally be as close to the backplate as possible. Another possibility
is to trim all of the LCD headers to make them just a little bit shorter. This
will make it even more difficult to align the pins when putting the case on,
but people have successfully repaired their thermostats by doing this.

Other causes of the touchscreen not registering taps include: the Raspberry Pi
was powered on when the LCD screen was connected improperly, someone pushed on
the screen too hard, or only some of the pins are fully connected to the
screen. If the touch part of the screen is broken, the above tests will all
have the same results (no response to taps). At this point, your options are to
replace the screen or just use the screen for viewing information and make all
the changes to the thermostat over wifi.

\subsubsection{Screen doesn't turn on}
If you see lights illuminated on the pi, but the screen doesn't turn on at all,
the screen is not receiving power, which means it is likely not connected
properly. Look in through the side of the case and see if you can tell if the
pins are aligned correctly. If you can not determine this, try removing the
shell of the case and connect the screen on without the shell of the case. The
screen should light up slightly when powered on. If this still doesn't work, it
is likely a defective screen. If you have another Raspberry Pi available, you
can connect it to the top left pins of the Pi (when the SD card is on the left)
and see if the screen lights up when that Pi is powered on with the HestiaPi's
SD card in it.

If the screen lights up slightly, but is blank, this is a different issue. In
this case, the screen is likely working and it's an issue with the software.
You should see the screen flicker within 60 seconds and you should see the
boot messages after a few minutes.

The fastest and easiest fix is to just copy the image onto the SD card again
and try again. If you are an advanced user and want to spend time
troubleshooting the issue to determine the root cause, we encourage you to
share whatever you find with the community at
\url{https://community.hestiapi.com}.

\subsubsection{Screen is slow to respond to taps}
The HestiaPi should respond to touches within a second. If it's responding
to taps, but it taking a long time, it's likely a software issue. In this
case, copying a fresh image onto the SD card will likely fix it. The other
thing that can be done to get it to respond faster to touches is to get the
fastest SD card available.

A video has been made to demonstrate how quickly the screen should respond to
taps. A stylus is used here to make it clear when the tap is occurring versus
when the screen is updating, but it should work the same when tapping with your
finger. \url{https://peertube.gsugambit.com/w/1AhKQByTrAg38HKSQUJFR3}

\subsubsection{How to edit files via SSH}
If you are very new to command line interface we would advise you taking a
short online course by searching for ``linux command line interface'' on your
favourite website.

To edit a file while you are inside SSH use the command

\texttt{sudo nano /path/to/your/file}

Then leave your mouse alone as it does not control you cursor anymore

Use only your keyboard and once you are done, press Ctrl+O to save and Ctrl+X to close.

\subsubsection{Start OpenHAB2 in Debug Mode}

For OpenHAB2 (v10.x image -- July 2018)
To monitor the OpenHAB logs without stopping the service run

\texttt{openhab-cli showlogs}

To start OpenHAB manually after stopping the service run

\texttt{openhab-cli start}

For older OpenHAB installations:
Stop OpenHAB first

\texttt{sudo service openhab2 stop}

and when it is stopped, start it manually

\texttt{/usr/share/openhab2/start\_debug.sh}

once (if) loaded type inside the OpenHAB session

\texttt{log:tail}

and notice any issues.
