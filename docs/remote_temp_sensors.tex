Sometimes the thermostat wires are poorly located. This may mean near a door
that throws off the temperature, in a room that gets a lot of sun and therefore
runs hotter than the other rooms, or any number of other problems. If this
sounds familiar, you may be interested in having a remote temperature sensor.

There are many different models of remote temperature sensors to choose from.
The main feature that you will want to look for when selecting a sensor is that
it supports MQTT. MQTT is the protocol that will allow the temperature sensor
to send the current temperature to HestiaPi.

To get the IP address of your HestiaPi, click the information icon (i) in the
top right of your HestiaPi. Consult the documentation of your temperature
sensor on how to configure it to publish temperature data to the HestiaPi. It
should be publishing to port 1883 on the HestiaPi, which is a standard port for
MQTT. There are some notes on how to set this up for Ruuvi tags in the section
on \ref{Ruuvi Temperature Sensors}.

After you've done this, the data is being sent to the HestiaPi, but we need to
reconfigure the thermostat to use this new data. This will be done via the web
interface, and specifically the PaperUI. You can get to this by navigating to
http://[YOUR-HESTIAPI-IP]:8080/paperui/index.html (where you insert the IP address of your
HestiaPi in labeled location).

We will walk through an example which sets up a temperature sensor in the living
room. You can use a different names for your sensor if appropriate.

First, we need to make an Item to store the data. Go to Configuration -> Items
and click the (+).  Enter the following settings, leaving anything not mentioned
at its default value:

\begin{itemize}
\item Name (can’t have spaces): livingroomtemp
\item Label (human readable, can have spaces): Living Room Temperature
\item Type: Number
\item Dimension: Temperature
\end{itemize} 

Click the plus in the blue circle at the top to save. For more information
about what an ``item'' is in OpenHAB terminology, see the OpenHAB documentation
at: \url{https://v2.openhab.org/docs/concepts/items.html}

Second, we need to make a Thing to represent the sensor. Go to Configuration ->
Things -> (+) -> MQTT Binding -> Add manually -> Generic MQTT Thing.

\begin{itemize}
\item Name (human readable, can have spaces): Living room sensor
\item Thing ID: livingroomsensor
\item Location: Sensors
\item Bridge selection: Mosquitto MQTT Broker
\end{itemize}

Click the checkmark in the blue circle at the top to save. For more information
about what a ``thing'' is in OpenHAB terminology, see the OpenHAB documentation
at: \url{https://v2.openhab.org/docs/concepts/things.html}

Third, we need to add a Channel to the Thing. We will just go over how to make
one channel for temperature, but additional channels could be made for humidity,
barometric pressure and anything else that your sensor can measure. Go to
Configuration -> Things -> Living room sensor -> Channels (+)

\begin{itemize}
\item Channel type: Number Value
\item Channel ID (no spaces): temp
\item Label (human readable): Temperature
\item MQTT State Topic: <see notes below>
\end{itemize}

The MQTT State Topic will depend on what topic your specific temperature sensor
is using when it is publishing the data. For example, the RuuviBridge will use
a prefix of ``ruuvi'' and then the BlueTooth MAC address of the Ruuvi tag. So
the MQTT State Topic would be something like ``ruuvi/AA:BB:CC:DD:EE:FF''.

Each MQTT topic \textbf{should} only have one value. For example, it would be
nice if the RuuviBridge published temperature data to the topic:
``ruuvi/AA:BB:CC:DD:EE:FF/temperature''. Then the humidity could go to
``ruuvi/AA:BB:CC:DD:EE:FF/humidity'' and so forth.  Unfortunately, that's not
how it currently works. Instead, RuuviBridge publishs all of the fields to a
single topic in JSON format. There are other temperarure sensors that also
follow this pattern.

To convert the entire JSON string to just the temperature data, click ``Show
More'' and enter ``JSONPATH:\$.temperature'' in the ``Incoming Value
Transformations'' field. This will extract just the temperature field. The
same can be done for ``humidity'' and ``pressure'' if you are interested in
making that information avilable to your thermostat.

If you configured the Channel to use the JSONPATH transformation, you'll need
to install that add-on. Go to Add-On -> Transformations tab, and click install
next to ``JSONPath Transformation''. It will take several minutes to install.
Please be patient.

Next, we want to create an Item to hold the temperature that is being recorded
by the temperature sensor on the HestiaPi, just in case we decide we want to
see it or use it in the future. Configuration -> Items -> (+)

\begin{itemize}
\item Name (can’t have spaces): localtemp
\item Label (human readable, can have spaces): Local Temperature
\item Category (optional, leave blank)
\item Type: String
\end{itemize}

Click the plus in the blue circle at the top to save.

Now we want to configure the local temperature sensor to send data to this new
Item and stop sending it to the ``MyTemp'' Item that is displayed on the
HestiaPi and is used to determine when to turn the heating and cooling on and
off. Go to Configuration -> Things -> Temperature Sensor

Under channels, click the blue concentric circles to the left of the ``Output''
channel, then click the trash can icon to the right of the linked item
``Temperature (MyTemp)''. This will cause the temperature on the HestiaPi to
stop updating (until we reconfigure it).

While still in the Output channel of the Temperature Thing: Next to Linked
Items -> (+)

\begin{itemize}
\item Profile: Default
\item Item to link: Local Temperature (localtemp)
\end{itemize}

Click the Link button to save. Now the temperature sensor on the HestiaPi
is going to this localtemp Item.

Now we need to change the type of the MyTemp Item from a String to a Number.
Go to Configuration -> Items -> Temperature (MyTemp) -> Pencil icon

\begin{itemize}
\item Type: Number
\end{itemize}

Click the check mark in the blue circle at the top to save.

It's important to leave the Dimension blank here. Setting it to Temperature
will result in an incorrect conversion from Celsius to Fahrenheit (so it would
display 32F when it is actually 0F).

Finally, we want to use a Channel to our new sensor Thing to the Item where the
data is stored. Specifically, we want our Living room sensor to update both the
livingroomsensor item as well as the MyTemp item.

Go to Configuration -> Things -> Living room sensor.

Under channels, click the blue concentric circles to the left of the
Temperature channel.

\begin{itemize}
\item Profile: Follow
\item Item to link: Temperature (MyTemp)
\end{itemize}

Then click on the concentric circles and edit the linked item.

\begin{itemize}
\item Profile: Default
\end{itemize}

The reason we have to choose Follow first is because MyTemp is not available
under the Default profile. However we can get to it under Follow and then
switch it back to the correct profile type of ``Default''.

Repeat this process to also link the Living room sensor to the ``Living Room
Temperature (livingroomtemp)'' item.

If you'd like to learn more about linking things and items with channels, the
OpenHAB documentation covers that here:
\url{https://v2.openhab.org/docs/configuration/paperui.html}

At this point, we should be all done. Go to Control -> SENSORS and you should
see the items you added (Local temperature, Living room temperature) and the
Temperature (which is what is used to actually control the HVAC) should match
the Living room temperature. The LCD on the HestiaPi should also be displaying
the Temperature, which should now be coming from the remote sensor.

Congratulations, you now have configured your HestiaPi to use your remote
temperature sensor!
